\section{Passing arguments to main.}\label{sec:premain}

CompCert can compile programs where \Ccode{main} takes arguments, but its correctness theorem gives no guarantees about their translation. That is because its semantic model assumes that \Ccode{main} takes no arguments (See \autoref{code:initial_state}); but real C programs can take up to two arguments \Ccode{argc}, the argument count, and \Ccode{argv}, the argument vector%(and \Ccode{envp}, the global environment)
. Also, all executions in the semantics of CompCert start with a call to \Ccode{main()}, even though \Ccode{main} is nothing but an agreed upon term for startup. We aim to generalize this to any function, not just main(). In this section we define a new predicate \Coqcode{entry_point}, generalizing \Coqcode{initial_state} (\autoref{code:initial_state}), that characterizes starting states which includes calls to \Ccode{main} with arguments and calls to any other function.

Passing arguments to the entry-function is particularly important for our work with concurrency because spawning new threads behaves much like starting a program by calling \Ccode{main}: The library function that spawns a new thread (e.g. \Ccode{pthread_create(foo)}) must create a new stack, push the arguments to stack and then call \Ccode{foo} just like a program initialization would do. Our new predicate \Coqcode{entry_point} is general enough to capture both process-start and thread-start, as well any kind of incoming function call that follows the appropriate calling convention.

It is worth pointing out how important it is to allow newly spawned functions to take arguments. If we restricted our semantics to spawning threads with no arguments, threads wouldn't be able to share pointers (or would have to do it clumsily through global variables) and thus they would all execute in disjoint pieces of memory with no communication. That would be a much easier and less interesting result. 

Once we define a new starting point for executions, we must prove that compilation preserves the predicate \Coqcode{entry_point} (\autoref{table:initial_sim}(b)) in the same way that CompCert's simulation preserves \Coqcode{initial_state} (\autoref{table:initial_sim}(a)). The proof largely follows the simulation of internal function calls which is already proven in CompCert, so we omit the details here. However, some interesting relevant details are presented later in \autoref{sec:entry_point}.

\begin{figure}\centering
\begin{multicols}{2}\begin{tikzcd}[column sep=tiny]
 \arrow[d, "\text{initial\_state}"']
&  &  \arrow[d, dotted,  "\text{initial\_state}"] \\
s_1 & \widesim{} & s_2
\end{tikzcd}

(a)

\begin{tikzcd}[column sep=tiny]
\arrow[d, "\text{entry\_points } "' near start, "m_0 \ f \ \text{args}"' near end]
&  &  \arrow[d, dotted,  "\text{entry\_points }"  near start, "m_0 \ f \ \text{args}"  near end] \\
s_1 & \widesim{} & s_2
\end{tikzcd}

(b)
\end{multicols}
\caption[Entry simulation diagrams]{Entry simulation diagrams: (a) if $s_1$ is an initial state for the source program, then there exists some state $s_2$ that is related to $s_1$ and is an initial state for the compiled program. (b) Just like the diagram for initial states, but it generalizes and exposes the initial memory $m_0$, the entry function $f$ and the arguments \Coqcode{args}. The entire simulation is parametric on the realation $\sim$.}\label{table:initial_sim}
\end{figure}

\subsection{The prestack and the initial memory}

When execution starts, CompCert semantics assumes that the stack is empty and the memory contains only the global variables. However, in reality, when \Ccode{main} starts executing, there is more content already pushed on the stack and in memory that is particularly important to argument passing. Part of the \Coqcode{entry_point} predicate is to describe this initial state of memory, as we explain below.

Let's take, as an example, the moment \main\ is called from the initialization function \Ccode{_libc_start_main}. At this point the top of the stack will contain the return address and the arguments to \main\  (beyond the first $K$ that are passed on registers; on some RISC machines, $K=4$, while on the x86-32, $K=0$). We call \emph{prestack} that tip of the stack, depicted in \autoref{figure:premain_stacks}(a), which is relevant to the execution of \main\ and does not correspond to any function in the program. The rest of the memory, at that point, also contains the \Ccode{NULL}-terminated argument vector, all the global variables, and possibly stacks of other threads or other initialization functions such as \Ccode{start}. We call the entire memory at this point the \emph{initial memory}. 
Our new predicate \Coqcode{entry_point}, instead of an empty stack, describes how arguments are set up in the prestack and, instead of an almost empty initial memory, allows memories to have arbitrary things. %This extra contents of memory can be the argument vector, stacks of other functions, stacks of other threads, or anything else.   



\begin{figure}\centering
\begin{multicols}{2}
%\vspace*{\fill}
%\begin{drawstacknoends}
%%\llcell{1}{highlight}{Stack frame for main}
%\startframe
%\cell{\Ccode{argc}}
%\cell{\Ccode{argv}} 
%\cell{\Ccode{envp}}
%\stackbottom[padding]
%\finishframe{Stack created \Ccode{exec} }
%\end{drawstacknoends}
%(a) Stack after calling \Ccode{exec()} to run a C program.
%
%\
%
%\columnbreak

\vspace*{\fill}

\begin{drawstacknoends}
\llcell{1}{freecell}{Stack frame for main}
%\startframe
\cell{return address}
\cell{\Ccode{argc}}
\cell{\Ccode{argv}} 
\cell{\Ccode{envp}}
\cell{\Ccode{esp}}
%\finishframe{\Ccode{_libc_start_main}}
\stackbottom[padding]
\end{drawstacknoends}

(a) Stack at the start of \main. 

\ 

\columnbreak

\vspace*{\fill}
\begin{drawstacknoends}
\llcell{2}{freecell}{Stack frame for \Ccode{foo}}
%\startframe
\cell{return address}
\cell{arg}
\cell{\Ccode{esp}}
\stackbottom[padding]
%\finishframe{\Ccode{pthread_create} }
\end{drawstacknoends}

(b) Stack created by \Ccode{pthread_create()} to start a thread running \Ccode{foo}. 

\end{multicols}
\caption[Prestacks examples for \Ccode{X86-32} architecture]{Prestacks examples for \Ccode{X86-32} architecture: (a) Stack right before \main\ executes. (b) Stack right before a function \Ccode{foo} is executed in a new thread. Notice that in X86-64 (or ARM, etc.), the first $K$ arguments will be passed in registers. }\label{figure:premain_stacks}
\end{figure}


% When an operating system creates a new process in which to run a C program, before main ever starts executing, the operating systems \Ccode{exec} procedure, sets up the stack, the memory and the registers, before \main\ is executed. The stack, at this point, contains \Ccode{argc}, \Ccode{argv} and \Ccode{envp} as depicted in \autoref{figure:premain_stacks}(a). The memory at this point also contains the \Ccode{NULL}-terminated argument vector, all the global variables and possibly more variables such as linker variable. Once that is done, the OS does not (in fact) invoke main, it invokes a function such as \Ccode{_start()} (sometimes \Ccode{premain} or \Ccode{startup}), which does things like aligning the stack and setting a final call to exit the process. In such a case, the prestack has even more things, as seen in \autoref{figure:premain_stacks}(b). 

%
%If \main\ 
%is invoked through a system call (e.g., \Ccode{execve}), as shown in \autoref{figure:premain_stacks}(a), the prestack contain the argument vector (if there are no arguments, it has a single element with a \Ccode{NULL} pointer), the argument count \Ccode{argc}, pointers to the arguments and the global environment, and possibly more variables such as linker variable. 
%
%The more common way to execute \Ccode{main} is to starts with an initialization routine, calling a function such as 
%
%Before \Ccode{main} ever starts executing, there is an initialization process that sets up the stack and the registers, before main is executed. The preprocessing function(s) often called \Ccode{_start}, \Ccode{premain} or \Ccode{startup}, will create a stack, set up the arguments in the stack or in registers, set up the environment (\Ccode{envp}), set up the return address (a call to \Ccode{exit()}), and other bookkeeping. 

As mentioned before, spawning a thread behaves like executing \main\ in many ways. For instance, the stack of a thread, before the spawned function executes, looks just like a prestack before calling \main\, as shown in \autoref{figure:premain_stacks}(b). Indeed, when \Ccode{pthread_create} starts a new thread, it sets up the stack to pass arguments to the spawned function. The initial memory, at this point, contains the stacks of other threads and all other memory used by their executions. \Coqcode{entry_point} is general enough to characterize this prestack and initial memory too.

%Reasoning about the prestack is hard because it's the only part in the stack that does not correspond to any function in the call stack of the program; it may corresponds to the stack frame of \Ccode{_start()}, which is part of the linked program. Since compilation might rearrange memory, we must be very careful about how the prestack is transformed. 
If only we could pass all arguments on registers, we wouldn't need to reason about the prestack at all. 
Unfortunately, in architectures such as $x86$ in $32$-bit mode, all arguments are passed on the stack. As the comments in the CompCert code put it, ``Snif!" \cite{leroy19:compcert}. Even architectures that allow argument passing in registers, such as $x86$ in $64$-bit mode, have a limited number of registers and will pass arguments on the stack after those run out. Consequently, if we want describe argument passing for entry functions in general, we must describe the prestack.

The  characterization of the prestack is language-dependent, and it will be described more in the next subsection. % (\autoref{sec:entry_point}).

\subsection{The \Ccode{entry_point}: a more permissive starting state}\label{sec:entry_point}
The predicate \Coqcode{entry_point: mem -> state -> val -> list val -> Prop} takes an initial memory \Coqcode{m}$_0$, an initial state \Coqcode{s}, a pointer to the entry function \Coqcode{fun_ptr} of type \Coqcode{val}, and a list of arguments \Coqcode{args}. This predicate is language dependent, and is divided in three parts:
\begin{enumerate}
\item Checks that the global environment \Coqcode{genv} is allocated correctly. It also makes sure that the pointer \Coqcode{fun_ptr} points to a function in \Coqcode{genv}.
\item Checks that memory  \Coqcode{m}$_0$ is well formed. That is, it contains no ill-formed pointers to invalid addresses. CompCert generally maintains that well-formed programs don't create dangling pointers.\footnote{A well-formed program, should not compare, read or write to invalid pointers. Hence, dangling pointers behave semantically as undefined values and could be modeled that way.}
\item Checks that arguments are well-formed. Among other things, they have the right types for the function being called, they have no ill-formed pointers, they fit in the stack and they correspond to the prestack.  
\end{enumerate}
In the rest of this section, we explore the definition of \Coqcode{entry_points} for different languages and, when interesting, we explain how we prove that different CompCert passes preserve the predicate as in \autoref{table:initial_sim}(b).

\subsubsection{C frontend}
All of the C-like languages (\emph{Clight, Csharp, Csharpminor}) have similar \Coqcode{entry_point}, so we present here the one for Clight in \autoref{code:entry_point_Clight}.
%\lstset{moredelim=*[s][\color{green}]{Inductive}{' '}} %playing arounds
\begin{figure}
\begin{lstlisting}[ numbers=left]
Inductive entry_point (ge:genv): mem -> state -> val -> list val -> Prop :=
| initi_core: forall f fb m0 args targs,
      let sg:= signature_of_type targs type_int32s cc_default in
      type_of_fundef (Internal f) = Tfunction targs type_int32s cc_default ->
      Genv.find_funct_ptr ge fb = Some (Internal f) ->
      globals_not_fresh ge m0 ->
      Mem.mem_wd m0 ->
      Val.has_type_list args (typlist_of_typelist targs) ->
      vars_have_type (fn_vars f) targs ->
      vals_have_type args targs ->
      Mem.arg_well_formed args m0 ->
      bounded_args sg ->
      entry_point ge m0 (Callstate (Internal f) args (Kstop targs) m0).
\end{lstlisting}
\caption{The \Ccode{entry_point} predicate in Clight}\label{code:entry_point_Clight}
\end{figure}
Lines 4-6 ensure that the environment is allocated in memory and it contains the function \Coqcode{f} with the right type signature. Line 7 states that the initial memory has no dangling pointers. Lines 8-11 say that the arguments have the right type and have no dangling pointers. The predicate \Coqcode{bounded-args}, enforces that the arguments fit in the stack, which is architecture dependent (generally around 1 Gigabyte). Finally, the entry state, in line 13, is defined as a call to \Coqcode{f} with an empty continuation.

In CompCert, continuations abstract the program's call stack with each \Coqcode{Kcall} describing a stackframe. Standard CompCert describes the ``stop" continuation as simply \Coqcode{Kstop} without an argument.  This does not permit description of a prestack frame containing arguments to main.  Consequently, during later phases of compilation, we cannot design a simulation relation that properly relates \Coqcode{Kstop} to the prestack frame with arguments allocated in memory.  It is precisely for this reason that we use \Coqcode{Kstop targs} instead of \Coqcode{Kstop}; from the types of the arguments \Coqcode{targs}, we can determine the (architecture-dependent) shape of the prestack. In our model, when \Coqcode{Kcall} gets translated to a predicate \Coqcode{Stackframe} that describes a stack frame, \Coqcode{Kstop} will be translated to \Coqcode{Prestack}, a special kind of \Coqcode{Stackframe}, as depicted in \autoref{fig:Kstop_compile}.

\newcolumntype{L}{>{\centering\arraybackslash}m{3.5cm}}

\begin{figure}
\begin{tabular}{ l L c L c L }
  (a) & $\text{\Coqcode{Kcall}}$&$  \longrightarrow  $&$\text{\Coqcode{Stackframe}}  $&$\longrightarrow$ & 
  \scalebox{0.50}{
\begin{drawstacknoends}%\startframe
\stacktop{}[padding]
\llcell{2}{freecell}{Stack frame}
%\finishframe{\Ccode{_libc_start_main}}
\stackbottom[padding]
\end{drawstacknoends}
} \\
  (b) & $\text{\Coqcode{Kstop}}$&$  \longrightarrow $&$ \text{\Coqcode{Prestack}}  $&$\longrightarrow$ & 
\scalebox{0.50}{
\begin{drawstacknoends}%\startframe
\stacktop{}[padding]
\llcell{2}{freecell}{prestack}
\end{drawstacknoends}
} \\
& Continuations in C-like languages  & & Stack frame descriptions in low level languages & & Stack frames in memory for Mach and Asm
\end{tabular}

%\begin{tikzcd}[column sep=tiny]
%%\arrow[d, "\text{entry\_points } m_0 \ f \ \text{args}"'] &  &  \arrow[d, dotted,  "\text{entry\_points } m_0 \ f \ \text{args}"] \\
%\text{(a)}&\text{\Coqcode{Kcall}} & \longrightarrow & \text{\Coqcode{Stackframe}} & \longrightarrow \\
%\text{(b)}&\text{\Coqcode{Kstop targs}} & \longrightarrow & \text{\Coqcode{Prestack}} & \longrightarrow  \\
%%\text{C-like languages} & & \text{Register transfer languages} & & \text{low level languages} \\
%\end{tikzcd}


\caption[Abstractions of stack frames through compilation]{Abstractions of stack frames become more concrete through compilation. (a) \Coqcode{Kcall} is a high level abstractions of stackframes for C-like languages. It gets translated to \Coqcode{Stackframes}, which are low level descriptions of stack frames. Finally the stack is laid in memory. (b) \Coqcode{Kstop}, is a concise, high level,  description of the the prestack. It gets compiled to \Coqcode{Prestack}, which is a special type of \Coqcode{Stackframe}, and finally laid in memory} \label{fig:Kstop_compile}
\end{figure}

%Our empty continuation \Coqcode{Kstop} takes \Coqcode{targs} as an argument. That is because continuations also represent the call stack; \Coqcode{Kstop targs} represents the preestack of \Ccode{_start()} that might contain some arguments for \Coqcode{f}.

\subsubsection{Register transfer languages}
In the Cminorgen phase, CompCert coalesces  all function variables into a stack frame. Some functions might get empty stack frames (i.e., a zero-sized memory block), if none of their variables has their address taken. These stack frames are important, even the empty ones, because that is where spill variables will be written after register allocation in the Allocation phase. We follow suit and create an empty stack frame for \Ccode{_start()}. The stack is empty because the compiler has not yet decided what arguments will be passed in memory and which ones in registers; even for architectures that pass all arguments in memory, this is not done until the Stacking pass. Hence, for languages between Cminorgen and Stacking (Cminor, CminorSel, RTL, LTL, Linear), there is an extra line in \Coqcode{entry_point} to make sure that the empty stack frame is allocated:
\begin{lstlisting}
	Mem.alloc m0 0 0 = (m1, stk) \end{lstlisting}
The rest of the predicate is almost identical to the one in \autoref{code:entry_point_Clight}.

These languages have a list of stack frame descriptors (called \Coqcode{Stackframe}) , instead of continuations; accordingly, the prestack is characterized by the predicate \Coqcode{Prestack}, a frame descriptor with just enough information to know the size of the prestack and where \main\ should return. \Coqcode{Prestack} is generated from \Coqcode{Kstop} and, after the \Coqcode{Stacking} pass, it is translated to an actual prestack as shown in \autoref{fig:Kstop_compile}.

\subsubsection{Machine languages}

In the machine languages (Mach and Asm), a function expects a certain shape from the stack frame of its caller. We replace the empty stack frame allocation with the construction of the prestack as shown in \autoref{code:entry_point_Mach}.
\begin{figure}
\begin{lstlisting}[ numbers=left]
      $\vdots$
      let '(stk_sz,ret_ofs,parent_ofs) := stack_defs (fn_sig f) in
      Mem.alloc m0 0 stk_sz = (m1, spb) ->
      let sp:= Vptr spb Ptrofs.zero in
      store_stack m1 sp Tptr parent_ofs Vnullptr = Some m2 ->
      store_stack m2 sp Tptr ret_ofs Vnullptr = Some m3 ->
      make_arguments (Regmap.init Vundef) m3 sp
                     (loc_arguments (funsig (Internal f))) args = Some (rs, m4) ->
      $\vdots$
\end{lstlisting}
\caption[\Ccode{entry_point} predicate in Mach]{Part of the \Ccode{entry_point} predicate in Mach}\label{code:entry_point_Mach}
\end{figure}
The function \Coqcode{stack-defs} is an architecture dependent function that calculates the layout of the stack and returns the size \Coqcode{stk_sz}, the offset of the return address \Coqcode{ret_ofs}, and a back link to parent frame \Coqcode{parent_ofs}. The last two values are unused, but the stack must have space for them. Line 3 allocates the stack of the correct size. Lines 5-8 store the return address, the link to the parent and the arguments in the stack. 

\subsubsection{Summary: entry points}
In summary, we have augmented the \compcert\ semantics to permit the entry point to be a function of more than zero arguments; in a memory that can contain more than just the extern initialized global variables.  These changes also required augmenting the \Coqcode{Kstop} continuation of the Clight language, to (abstractly) describe the initial stack frame.


















