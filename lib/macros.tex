%Copied from our LICS submission (taxonomy)

%%%%%%%%%%%%%%
%% Separation macros
\newcommand{\wand}{{-  \mkern-6mu*}}
\newcommand{\emp}{{\keyw{emp}}}

%Compcertpermissions
\newcommand{\Free}{{\keyw{Freeable}}}
\newcommand{\Write}{{\keyw{Writable}}}
\newcommand{\Read}{{\keyw{Readable}}}
\newcommand{\Nempty}{{\keyw{Nonempty}}}
\newcommand{\nonpos}{{\keyw{nonpos}}}
%\newcommand{\None}{{\mathrm{None}}}


%%%%%%%%%%%%%%%%%%%%%%%%%%%%%%%%%%%%%%%%%%%%%%%%%%%%%%%%%%%%%%%%%%%%%%%
%% Rustan wants the letters in his math formulas to be spaced
%% as in running italic text.  So we here redefine each letter,
%% in its role as a MathSymbol, to come from the symbol font
%% named "italics", rather than from the one named "letters".

% The following line is needed to make this file work with certain
% fonts, include the standard Computer Modern fonts.
\DeclareSymbolFont{italics}{OT1}{cmr}{m}{it}

\DeclareMathSymbol{a}{\mathalpha}{italics}{"61}
\DeclareMathSymbol{b}{\mathalpha}{italics}{"62}
\DeclareMathSymbol{c}{\mathalpha}{italics}{"63}
\DeclareMathSymbol{d}{\mathalpha}{italics}{"64}
\DeclareMathSymbol{e}{\mathalpha}{italics}{"65}
\DeclareMathSymbol{f}{\mathalpha}{italics}{"66}
\DeclareMathSymbol{g}{\mathalpha}{italics}{"67}
\DeclareMathSymbol{h}{\mathalpha}{italics}{"68}
\DeclareMathSymbol{i}{\mathalpha}{italics}{"69}
\DeclareMathSymbol{j}{\mathalpha}{italics}{"6A}
\DeclareMathSymbol{k}{\mathalpha}{italics}{"6B}
\DeclareMathSymbol{l}{\mathalpha}{italics}{"6C}
\DeclareMathSymbol{m}{\mathalpha}{italics}{"6D}
\DeclareMathSymbol{n}{\mathalpha}{italics}{"6E}
\DeclareMathSymbol{o}{\mathalpha}{italics}{"6F}
\DeclareMathSymbol{p}{\mathalpha}{italics}{"70}
\DeclareMathSymbol{q}{\mathalpha}{italics}{"71}
\DeclareMathSymbol{r}{\mathalpha}{italics}{"72}
\DeclareMathSymbol{s}{\mathalpha}{italics}{"73}
\DeclareMathSymbol{t}{\mathalpha}{italics}{"74}
\DeclareMathSymbol{u}{\mathalpha}{italics}{"75}
\DeclareMathSymbol{v}{\mathalpha}{italics}{"76}
\DeclareMathSymbol{w}{\mathalpha}{italics}{"77}
\DeclareMathSymbol{x}{\mathalpha}{italics}{"78}
\DeclareMathSymbol{y}{\mathalpha}{italics}{"79}
\DeclareMathSymbol{z}{\mathalpha}{italics}{"7A}

\DeclareMathSymbol{A}{\mathalpha}{italics}{"41}
\DeclareMathSymbol{B}{\mathalpha}{italics}{"42}
\DeclareMathSymbol{C}{\mathalpha}{italics}{"43}
\DeclareMathSymbol{D}{\mathalpha}{italics}{"44}
\DeclareMathSymbol{E}{\mathalpha}{italics}{"45}
\DeclareMathSymbol{F}{\mathalpha}{italics}{"46}
\DeclareMathSymbol{G}{\mathalpha}{italics}{"47}
\DeclareMathSymbol{H}{\mathalpha}{italics}{"48}
\DeclareMathSymbol{I}{\mathalpha}{italics}{"49}
\DeclareMathSymbol{J}{\mathalpha}{italics}{"4A}
\DeclareMathSymbol{K}{\mathalpha}{italics}{"4B}
\DeclareMathSymbol{L}{\mathalpha}{italics}{"4C}
\DeclareMathSymbol{M}{\mathalpha}{italics}{"4D}
\DeclareMathSymbol{N}{\mathalpha}{italics}{"4E}
\DeclareMathSymbol{O}{\mathalpha}{italics}{"4F}
\DeclareMathSymbol{P}{\mathalpha}{italics}{"50}
\DeclareMathSymbol{Q}{\mathalpha}{italics}{"51}
\DeclareMathSymbol{R}{\mathalpha}{italics}{"52}
\DeclareMathSymbol{S}{\mathalpha}{italics}{"53}
\DeclareMathSymbol{T}{\mathalpha}{italics}{"54}
\DeclareMathSymbol{U}{\mathalpha}{italics}{"55}
\DeclareMathSymbol{V}{\mathalpha}{italics}{"56}
\DeclareMathSymbol{W}{\mathalpha}{italics}{"57}
\DeclareMathSymbol{X}{\mathalpha}{italics}{"58}
\DeclareMathSymbol{Y}{\mathalpha}{italics}{"59}
\DeclareMathSymbol{Z}{\mathalpha}{italics}{"5A}
%% end Rustan's math fonts

%-----------------compact itemization ------------------------------------
\newenvironment{ditemize}{%   itemized lists w/o interline space
\begin{list}{$\bullet$}{%
\setlength{\itemsep}{0pt}\setlength{\rightmargin}{0pt}%
\setlength{\leftmargin}{1em}\setlength{\parsep}{0ex}}}{
\end{list}}
%-------------------------------------------------------------------------

%%% Bold math alphabet
\DeclareMathAlphabet{\mathbfsf}{\encodingdefault}{\sfdefault}{bx}{n}

%%% Basic macros

\newcommand{\etc}{\emph{etc.}}
\newcommand{\etal}{\emph{et al.}}
\newcommand{\eg}{\emph{e.g.}}
\newcommand{\ie}{\emph{i.e.}}

\newcommand{\keyw}[1]{\ensuremath{\mathsf{#1}}\xspace}
\newcommand{\option}[1]{\keyw{option}\;#1}
\newcommand{\List}[1]{\keyw{list}\;#1}
\newcommand{\Set}[1]{\keyw{set}\;#1}
\newcommand{\Type}{\keyw{Type}}
\newcommand{\Prop}{\keyw{Prop}}
\newcommand{\Some}[1]{\keyw{Some}\ #1}
\newcommand{\None}{\keyw{None}}

\newcommand{\blueboldkeyw}[1]{{\color{MidnightBlue} \mathbfsf{{#1}}}}
\newcommand{\blackboldkeyw}[1]{{\color{Black} \mathbfsf{{#1}}}}

\newcommand{\myif}[0]{\blackboldkeyw{if}}
\newcommand{\mythen}[0]{\blackboldkeyw{then}}
\newcommand{\myelse}[0]{\blackboldkeyw{else}}
\newcommand{\mycase}[0]{\blackboldkeyw{case}}
\newcommand{\myof}[0]{\blackboldkeyw{of}}
\newcommand{\myis}[0]{\blackboldkeyw{is}}
\newcommand{\mywith}[0]{\blackboldkeyw{with}}
\newcommand{\mylet}[0]{\blackboldkeyw{let}}
\newcommand{\myin}[0]{\blackboldkeyw{in}}
\newcommand{\myfieldupd}[3]{#1\ \mywith\ \{#2:=#3\}}
\newcommand{\mydo}[0]{\blackboldkeyw{do}}
\newcommand{\myreturn}[0]{\blackboldkeyw{return}}
\newcommand{\mydefault}[0]{\_\!\_\!\_}

\newcommand{\codecomment}[1]{
  \textrm{/\!/}\textrm{\ #1}
}

\newcommand\pair[2]{\left<#1,#2\right>}
\newcommand{\boxdotright}{\!\mathrel\boxdot\joinrel\rightarrow\!}
\newcommand{\islock}{\boxdotright}
\newcommand{\islocksh}[1]{\stackrel{#1}\boxdotright}
\newcommand{\nxt}[2]{\mathsf{next}\,#1\,#2}
\newcommand{\lseg}[2]{#1\!\! \leadsto \!\!#2}

\renewcommand{\to}{\rightarrow}
\newcommand{\topart}{\rightharpoonup}
\newcommand{\defeq}{\triangleq}

\newcommand{\upd}[3]{#1[#2\mapsto #3]}
\newcommand{\eval}[6]{#1 \vdash #5 \Downarrow_{#2, #3, #4} #6}
\newcommand{\step}[5]{#1 \vdash #2, #3 \longmapsto #4, #5}
%%% \estep: #6 is effect set
\newcommand{\estep}[6]{#1 \vdash #2, #3\ \overset{#6}{\longmapsto} #4, #5}
\newcommand{\tightoverset}[2]{%
  \mathop{#2}\limits^{\vbox to -.5ex{\kern-0.6ex\hbox{$\hspace{-7pt}\scriptstyle{#1}$}\vss}}}
\newcommand{\estepplus}[6]{#1 \vdash #2, #3\ \tightoverset{#6}{\longmapsto^{\!+}} #4, #5}
\newcommand{\estepnoge}[5]{#1, #2\ \overset{#5}{\longmapsto} #3, #4}
\newcommand{\estepplusnoge}[5]{#1, #2\ \tightoverset{#5}{\longmapsto^{\!+}} #3, #4}
%%% \rcstep: same args as \estep
\newcommand{\rcstep}[6]{#1 \vdash #2, #3\ \overset{#6}{\longmapsto}_{\keyw{rc}} #4, #5}

\newcommand{\RelatedText}[5][0.425\textwidth]{%
\begin{math}%
  \left.% 
  \parbox{#1}{#4}\vphantom{\parbox{#1}{#5}}%
  \right#3%
  \parbox{#2}{#5}\vphantom{\parbox{#2}{#4}}%
\end{math}
}%

\newcommand{\interp}[1]{\llbracket #1 \rrbracket}

%%% Core semantics syntax

\newcommand{\valtype}{\mathcal{V}}
\newcommand{\eftype}{\mathcal{F}}
\newcommand{\initCore}{\keyw{initial\_core}}
\newcommand{\atExt}{\keyw{at\_external}}
\newcommand{\aftExt}{\keyw{after\_external}}
\newcommand{\halted}{\keyw{halted}}
\newcommand{\corestep}{\keyw{corestep}}

%%% Linking macros

\newcommand{\link}[1]{\mathcal{L}(#1)}
\newcommand{\concur}[1]{\mathcal{C}(#1)}
\newcommand{\context}[2]{\link{#1, #2}}

%%% CompCert macros

\newcommand{\Vptr}{\keyw{Vptr}}
\newcommand{\mem}{\keyw{mem}}
\newcommand{\Genv}{\keyw{Genv}}
\newcommand{\val}{\keyw{val}}

%%% Structured injection macros

\newcommand{\Ownership}{\keyw{Ownership}}
\newcommand{\Own}{\keyw{own}}
\newcommand{\StrInj}{\keyw{StructuredInjection}}
\newcommand{\us}{\keyw{f_{us}}}
\newcommand{\them}{\keyw{f_{them}}}
\newcommand{\all}{\keyw{f_{all}}}
\newcommand{\vis}{\keyw{vis}}
\newcommand{\internIncr}{\sqsubseteq_{\keyw{us}}}
\newcommand{\externIncr}{\sqsubseteq_{\keyw{them}}}
\newcommand{\owned}{\keyw{owned}\xspace}
\newcommand{\shared}{\keyw{shared}\xspace}
\newcommand{\foreign}{\keyw{foreign}\xspace}
\newcommand{\public}{\keyw{public}\xspace}
\newcommand{\publicize}{\keyw{export}\xspace}
\newcommand{\syndicate}{\keyw{import}\xspace}
\newcommand{\inject}{\keyw{inject}}
\newcommand{\presglobs}{\keyw{preserves\_globals}}

%%% Structured simulation macros

\newcommand{\injects}[3]{#2 \rightarrowtail_{#1} #3}
\newcommand{\ssim}{\preceq}
%%% \match parameters:
%%% 1. \mu 
%%% 2. c_src, 3. m_src
%%% 4. c_tgt, 5. m_tgt
\newcommand{\match}[5]{
  \langle #2, #3\rangle \sim_{#1} \langle #4, #5\rangle }

\newcommand{\leakout}{\keyw{leak\_out}}
\newcommand{\boldleakout}{\blackboldkeyw{leak\_out}}

\newcommand{\leakin}{\keyw{leak\_in}}
\newcommand{\boldleakin}{\blackboldkeyw{leak\_in}}

\newcommand{\unchon}{\keyw{unchanged\_on}}
\newcommand{\boldunchon}{\blackboldkeyw{unchanged\_on}}

\newcommand{\blocksOf}{\keyw{blocksOf}}

%%% Reach-closed macros
\newcommand{\RC}[1]{#1_{\keyw{rc}}}

\newcommand{\restrict}[2]{\ensuremath{{#1}\!\downharpoonright_{#2}}}

%%% CompComp macros
\newcommand{\phase}[1]{#1}


\newcommand{\later}{\triangleright}
